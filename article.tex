\documentclass[12pt,a4paper]{article}
\usepackage{amsmath}
\usepackage{hyperref}
\usepackage{geometry}
\geometry{margin=1in}

\title{A New Gravitational Model Incorporating Pressure Effects: Numerical Analysis Across Celestial Bodies}
\author{Mohammadreza Eskandari Nasab Siahkohi \\
\textit{In collaboration with ChatGPT (AI Assistant)}}
\date{}

\begin{document}

\maketitle

\begin{abstract}
This paper presents a revised gravitational field model incorporating internal pressure components—including hydrostatic, rotational, and radiative pressures—in the computation of surface gravitational acceleration. Using real data from planets, the Sun, neutron stars, and galaxies, the influence of internal pressure on gravitational acceleration is quantified and compared with classical values. Results demonstrate that the modified model explains certain observed deviations in gravitational measurements and may provide insights into dark matter effects on galactic scales.
\end{abstract}

\section{Introduction}
Gravity is the dominant force shaping the structure and evolution of celestial bodies. Traditional gravitational models consider only total mass and its distribution, often neglecting internal pressures arising from complex physical conditions. However, in dense and massive bodies such as neutron stars, or interiors of stars and planets, these pressures can significantly influence the local gravitational field. This study investigates the role of internal pressure in modifying gravitational acceleration across various astrophysical scales. This approach could improve understanding of phenomena such as neutron star stability, planetary structure, and galactic dynamics influenced by dark matter.

\section{Theoretical Framework and Methodology}
\subsection{Model Assumptions}
\begin{itemize}
    \item Celestial bodies are approximated as uniform spheres.
    \item Internal pressures considered include hydrostatic, rotational, and radiative components.
    \item Magnetic fields and nonlinear dynamic phenomena are excluded from this first-order model.
\end{itemize}

\subsection{Model Equations}
The classical surface gravity is given by:
\[
g = \frac{GM}{R^{2}}
\]

The pressure correction term is:
\[
\Delta g = \frac{1}{\rho R} \sum_i P_i
\]

Thus, the modified gravitational acceleration is:
\[
g_{\text{model}} = g - \Delta g
\]

Where: \\
$G$ = gravitational constant, \\
$M$ = total mass, \\
$R$ = radius, \\
$\rho$ = average density, \\
$P_i$ = internal pressure components.

\section{Numerical Calculations and Comparison}
\subsection{Earth}
\[
\sum P_i = 1.202 \times 10^{9} \text{ Pa}, \quad \rho = 5514 \text{ kg/m}^3, \quad R = 6.371 \times 10^{6} \text{ m}
\]
\[
\Delta g = 0.0342 \text{ m/s}^2, \quad g_{\text{model}} = 9.7858 \text{ m/s}^2
\]

\subsection{Sun}
\[
\sum P_i = 2.5 \times 10^{13} \text{ Pa}, \quad \rho = 1408 \text{ kg/m}^3, \quad R = 6.957 \times 10^{8} \text{ m}
\]
\[
\Delta g \approx 25.54 \text{ m/s}^2, \quad g_{\text{model}} = 248.46 \text{ m/s}^2
\]

\subsection{Mars}
\[
\sum P_i = 3.505 \times 10^{8} \text{ Pa}, \quad \rho = 3933 \text{ kg/m}^3, \quad R = 3.39 \times 10^{6} \text{ m}
\]
\[
\Delta g = 0.0263 \text{ m/s}^2, \quad g_{\text{model}} = 3.6837 \text{ m/s}^2
\]

\subsection{Jupiter}
\[
\sum P_i = 7.531 \times 10^{10} \text{ Pa}, \quad \rho = 1326 \text{ kg/m}^3, \quad R = 6.991 \times 10^{7} \text{ m}
\]
\[
\Delta g = 0.812 \text{ m/s}^2, \quad g_{\text{model}} = 23.978 \text{ m/s}^2
\]

\subsection{Neutron Star}
\[
\sum P_i = 1.055 \times 10^{18} \text{ Pa}, \quad \rho = 7.0 \times 10^{17} \text{ kg/m}^3, \quad R = 1.2 \times 10^{4} \text{ m}
\]
\[
\Delta g = 1.25 \times 10^{-4} \text{ m/s}^2, \quad g_{\text{standard}} \approx 9.27 \times 10^{11} \text{ m/s}^2
\]

\subsection{Galaxy Scale}
\[
P = 1.0 \times 10^{-11} \text{ Pa}, \quad \rho = 10^{-21} \text{ kg/m}^3, \quad R = 5 \times 10^{20} \text{ m}
\]
\[
\Delta g = 2.0 \times 10^{-10} \text{ m/s}^2, \quad g_{\text{standard}} \approx 2.67 \times 10^{-10} \text{ m/s}^2
\]

\section{Discussion and Conclusion}
The pressure-inclusive gravitational model reveals notable corrections in gravitational acceleration across different celestial scales. In extremely dense objects such as neutron stars, pressure corrections are negligible, while in planets and stars they are significant. On a galactic scale, the correction magnitude is comparable to the gravitational acceleration itself, suggesting a potential connection with observed dark matter effects. This model offers a promising tool for deeper astrophysical and gravitational studies, especially for phenomena unexplained by classical Newtonian or relativistic frameworks.

\section{Future Work}
\begin{itemize}
    \item Integrating strong magnetic fields and electromagnetic radiation effects.
    \item Investigating implications for planetary and galactic evolution.
    \item Applying the model to multi-body simulations and observational datasets.
    \item Expanding to general relativistic corrections and energy-momentum tensor formulations.
\end{itemize}

\section{Acknowledgments}
The authors gratefully acknowledge the analytical and conceptual contributions of \textbf{ChatGPT}, which assisted with the theoretical formulation, equation verification, and numerical analysis of the model. The collaboration between human insight and AI computing power has enabled the development of this innovative gravitational framework.

\section{References}
\begin{enumerate}
    \item NASA Planetary Fact Sheets – \url{https://nssdc.gsfc.nasa.gov/planetary/planetfact.html}
    \item European Space Agency (ESA), Solar System Data – \url{https://www.esa.int/Science_Exploration/Space_Science}
    \item Carroll, B. W., \& Ostlie, D. A. (2017). \textit{An Introduction to Modern Astrophysics}. Cambridge University Press.
    \item Misner, C. W., Thorne, K. S., \& Wheeler, J. A. (1973). \textit{Gravitation}. W.H. Freeman and Company.
    \item Shapiro, S. L., \& Teukolsky, S. A. (1983). \textit{Black Holes, White Dwarfs, and Neutron Stars: The Physics of Compact Objects}. Wiley-VCH.
    \item Planck Collaboration (2020). \textit{Planck 2018 results. VI. Cosmological parameters}. \textit{Astronomy \& Astrophysics}, 641, A6.
    \item Peebles, P. J. E. (2020). \textit{Dark Matter and the Structure of the Universe}. \textit{Nature Astronomy}, 4, 725–729.
    \item Binney, J., \& Tremaine, S. (2008). \textit{Galactic Dynamics} (2nd ed.). Princeton University Press.
    \item Open Exoplanet Catalog, Average Densities and Pressures of Celestial Bodies – \url{https://www.openexoplanetcatalogue.com}
    \item Freedman, R. A., \& Kaufmann, W. J. (2015). \textit{Universe}. W.H. Freeman.
\end{enumerate}

\end{document}
